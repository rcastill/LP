\documentclass{beamer}
\usetheme{Madrid}
\useoutertheme{infolines}
\beamertemplatenavigationsymbolsempty
\usepackage[utf8]{inputenc}
\usepackage{ragged2e} % \justifying
\usepackage{graphicx}
\usepackage{listings}
\graphicspath{{\files}}

% Default fixed font does not support bold face
\DeclareFixedFont{\ttb}{T1}{txtt}{bx}{n}{12} % for bold
\DeclareFixedFont{\ttm}{T1}{txtt}{m}{n}{12}  % for normal

% Custom colors
\usepackage{color}
\definecolor{deepblue}{rgb}{0,0,0.5}
\definecolor{deepred}{rgb}{0.6,0,0}
\definecolor{deepgreen}{rgb}{0,0.5,0}

\usepackage{upquote}
\usepackage{listings}

\lstset{mathescape,basicstyle=\ttfamily,breaklines=true,showstringspaces=false,literate={á}{{\'a}}1{é}{{\'e}}1{í}{{\'i}}1{ó}{{\'o}}1{ú}{{\'u}}1{ü}{{\"u}}1}

% Python style for highlighting
\newcommand\pythonstyle{\lstset{
language=Python,
basicstyle=\ttfamily,
otherkeywords={self},             % Add keywords here
keywordstyle=\ttb\color{deepblue},
emph={MyClass,__init__},          % Custom highlighting
emphstyle=\ttb\color{deepred},    % Custom highlighting style
stringstyle=\color{deepgreen},
showstringspaces=false            % 
}}

% Python environment
\lstnewenvironment{python}[1][]
{
\pythonstyle
\lstset{#1}
}
{}

% Python for external files
\newcommand\pythonexternal[2][]{{
\pythonstyle
\lstinputlisting[#1]{#2}}}

% Python for inline
\newcommand\pythoninline[1]{{\pythonstyle\lstinline!#1!}}

% Provide translations for blocks and primitives
\uselanguage{spanish}
\languagepath{spanish}
\deftranslation[to=spanish]{Example}{Ejemplo}
\deftranslation[to=spanish]{Definition}{Definición}

\title[Lingua Franca]{Ayudantía Lenguajes de Programación}
\subtitle{\emph{Lingua Franca}}
\author[Rodolfo Castillo Mateluna]{\small Rodolfo Castillo Mateluna}
\institute[DI UTFSM]{U. Técnica Federico Santa María\\
Departamento de Informática}
\date{Segundo semestre 2017}

\setbeamertemplate{blocks}[default]
\setbeamertemplate{itemize items}[circle]
\setbeamertemplate{section in toc}{\inserttocsection}
\setbeamertemplate{subsection in toc}[square]

\begin{document}
\makeatletter
\setbeamertemplate{footline}
{
  \leavevmode%
  \hbox{%
  \begin{beamercolorbox}[wd=.333333\paperwidth,ht=2.25ex,dp=1ex,center]{author in head/foot}%
    \usebeamerfont{author in head/foot}\insertshortauthor\expandafter\beamer@ifempty\expandafter{\beamer@shortinstitute}{}{~~(\insertshortinstitute)}
  \end{beamercolorbox}%
  \begin{beamercolorbox}[wd=.333333\paperwidth,ht=2.25ex,dp=1ex,center]{title in head/foot}%
    \usebeamerfont{title in head/foot}\insertshorttitle
  \end{beamercolorbox}%
  \begin{beamercolorbox}[wd=.333333\paperwidth,ht=2.25ex,dp=1ex,right]{date in head/foot}%
    \usebeamerfont{date in head/foot}\insertshortdate{}\hspace*{2em}
    \insertframenumber{} / \inserttotalframenumber
    \hspace*{1em}
  \end{beamercolorbox}}%
  \vskip0pt%
}
\makeatother

	\begin{frame}[noframenumbering,plain]
		\titlepage
	\end{frame}
	
%	\begin{frame}[noframenumbering,plain]{Contenidos}
%		\tableofcontents
%	\end{frame}
	
	\begin{frame}{Track list}
		\begin{enumerate}
			\item Entry point
			\item Hello World!
				\begin{itemize}
					\item Manpage (\texttt{printf})
				\end{itemize}
			\item Compilación básica (\texttt{gcc})
			\item Variables (\emph{diapos profesor})
			\item Funciones
			\item \texttt{typedef}
		\end{enumerate}
	\end{frame}
	
	\begin{frame}{Tipos de datos}
	\begin{block}{Primitivos}
		\begin{itemize}
			\item Tipos básicos
			\item \texttt{sizeof}
			\item \textbf{En C nunca asuma nada excepto que \texttt{sizeof(char) == 1}}
		\end{itemize}
	\end{block}
	
	\begin{block}{Structs}
		\begin{itemize}
			\item Memoria contigua!
			\item \texttt{typedef}
			\item Structs anidados
			\item Structs anónimos
			\item Notación (.) y (-\textgreater)
		\end{itemize}
	\end{block}
	\end{frame}
	
	\begin{frame}{Tipos de datos}
	\begin{block}{Unions}
		\begin{itemize}
			\item Memoria compartida!
			\item Analogía con structs
			\item Strongly typed? (\emph{ver wiki})
		\end{itemize}
	\end{block}
	
	\begin{block}{Array}
		\begin{itemize}
			\item Declaración y definición
			\item Qué es realmente un \emph{array}?
		\end{itemize}
	\end{block}
	
	\begin{block}{Strings}
		\begin{itemize}
			\item Un \emph{array} particular
		\end{itemize}
	\end{block}
	\end{frame}
	
	\begin{frame}{Tipos de datos: Punteros! \tiny(\emph{you either love them or... you'll be forced to use them anyway :D)}}
		\begin{itemize}
			\item Son una variable como cualquier otra! (no se confunda innecesariamente)
			\item Aritmética (y magia negra del estilo)
			\item Visualización de \emph{layout} y arquitectura de HW
			\item \texttt{malloc}, \texttt{free} e importancia de \texttt{sizeof}
			\item Punteros a función
			\begin{itemize}
				\item \texttt{typedef}
				\item Ejemplo: \emph{Pipeline}
			\end{itemize}
		\end{itemize}
	\end{frame}
	
    \begin{frame}{Anexo}
    	\begin{itemize}
    		\item Makefile
    		\item Modularización
    		\begin{itemize}
	    		\item \emph{Header guard}
    		\end{itemize}
    	\end{itemize}
    \end{frame}
    
\end{document}












