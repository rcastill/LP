\documentclass[pdf]{beamer}
\usetheme{Rochester}
\beamertemplatenavigationsymbolsempty
\usepackage[utf8]{inputenc}
\usepackage{ragged2e} % \justifying

% Default fixed font does not support bold face
\DeclareFixedFont{\ttb}{T1}{txtt}{bx}{n}{12} % for bold
\DeclareFixedFont{\ttm}{T1}{txtt}{m}{n}{12}  % for normal

% Custom colors
\usepackage{color}
\definecolor{deepblue}{rgb}{0,0,0.5}
\definecolor{deepred}{rgb}{0.6,0,0}
\definecolor{deepgreen}{rgb}{0,0.5,0}

\usepackage{upquote}
\usepackage{listings}

\lstset{basicstyle=\ttfamily,breaklines=true,showstringspaces=false}

% Python style for highlighting
\newcommand\pythonstyle{\lstset{
language=Python,
basicstyle=\ttfamily,
otherkeywords={self},             % Add keywords here
keywordstyle=\ttb\color{deepblue},
emph={MyClass,__init__},          % Custom highlighting
emphstyle=\ttb\color{deepred},    % Custom highlighting style
stringstyle=\color{deepgreen},
showstringspaces=false            % 
}}

% Python environment
\lstnewenvironment{python}[1][]
{
\pythonstyle
\lstset{#1}
}
{}

% Python for external files
\newcommand\pythonexternal[2][]{{
\pythonstyle
\lstinputlisting[#1]{#2}}}

% Python for inline
\newcommand\pythoninline[1]{{\pythonstyle\lstinline!#1!}}

% Provide translations for blocks and primitives
\uselanguage{spanish}
\languagepath{spanish}
\deftranslation[to=spanish]{Example}{Ejemplo}
\deftranslation[to=spanish]{Definition}{Definición}

\title{Ayudantía Lenguajes de Programación}
\subtitle{PLY: Python Lex-Yacc}
\author{Rodolfo Castillo Mateluna}
\institute{U. Técnica Federico Santa María}
\date{Segundo semestre académico, 2017}

\begin{document}
	\begin{frame}
		\titlepage
	\end{frame}
	
	
	\begin{frame}{Disclaimer}
	\justifying
		Gran parte del contenido de esta presentación está basado en la documentación oficial de \texttt{PLY}.
		
		\medskip
		Esta puede ser encontrada en \href{www.dabeaz.com/ply/ply.html}{www.dabeaz.com/ply/ply.html}
	\end{frame}
	
	
	\begin{frame}{Requerimientos}
	\begin{itemize}
		\item Python (\textgreater\ 2.6)
		\item Conocimiento general de \textbf{gramáticas formales}
		\item Expresiones regulares
	\end{itemize}
	\end{frame}
	
	\begin{frame}{Lex}
		\begin{definition}
			Un \emph{lexer} o ``analizador léxico" (crudamente traducido al español) es un programa que ejecuta el proceso de convertir una secuencia de caracteres en una secuencia de \emph{strings} con un significado asociado.
		\end{definition}
		
		\begin{itemize}
			\item\texttt{Lex} es un programa que genera \emph{lexers}. Es originalmente una librería con \emph{bindings} para el lenguaje de programación \texttt{C}.
			\item Comunmente es acompañado de \texttt{Yacc}
		\end{itemize}
	\end{frame}
	
	\begin{frame}{Yacc (\emph{Yet Another Compiler-Compiler})}
		\begin{definition}
			\emph{Parsing} o ``análisis sintáctico" es el proceso de analizar un \emph{string} que sigue las reglas de una \textbf{gramática formal}
		\end{definition}
	\end{frame}
	
	\begin{frame}{Yacc (\emph{Yet Another Compiler-Compiler})}
		\begin{definition}
			La \textbf{derivación} (\emph{derivation}) de un \emph{string} es la secuencia de reglas de una gramática formal aplicadas para transformar el símbolo inicial en el \emph{string}
		\end{definition}
		\justifying
		Más adelante veremos que es muy importante tener en consideración si el proceso de análisis sintáctico se realiza como una \emph{leftmost} o \emph{rightmost derivation}
	\end{frame}
	
	\begin{frame}{Yacc (\emph{Yet Another Compiler-Compiler})}
		\begin{definition}
			Un \emph{LALR parser} (\emph{Look ahead LR parser}) es un tipo de analizador sintáctico optimizado en base a un \emph{LR parser} (\emph{Left to right, rightmost derivation})
		\end{definition}
	\end{frame}
	
	\begin{frame}{Yacc (\emph{Yet Another Compiler-Compiler})}
		\textbf{Yacc} es un \emph{LALR parser generator}. En otras palabras, es un generador de \emph{parsers}. Un \emph{parser} es un programa que realiza análisis sintáctico. Entonces, es correcto decir que Yacc, es un programa que genera un programa para analizar \emph{strings} potencialmente pertenecientes a un lenguaje generado por una gramática formal.
	\end{frame}
	
	\begin{frame}{Yacc (\emph{Yet Another Compiler-Compiler})}
		\textbf{Yacc} es un \emph{LALR parser generator}. En otras palabras, es un generador de \emph{parsers}. Un \emph{parser} es un programa que realiza análisis sintáctico. Entonces, es correcto decir que Yacc, es un programa que genera un programa para analizar \emph{strings} potencialmente pertenecientes a un lenguaje generado por una gramática formal.
		
		\medskip
		\centering
		¡Ahora \emph{Compiler Compiler} tiene sentido!
	\end{frame}
	
	\begin{frame}{A lo que nos convoca}
		\begin{itemize}
			\item PLY es una librería que implementa Lex+Yacc en Python
			\item Inicialmente fue desarrollada para un curso de compiladores
			\item PLY utiliza \textbf{reflexión} para generar \emph{lexers} y \emph{parsers}
		\end{itemize}
	\end{frame}
	
	\begin{frame}{\texttt{lex.py}}
		Conociendo la definición de un \emph{lexer} y la funcionalidad del programa \emph{Lex}, a continuación se mostrarán ejemplos de uso de su implementación en \emph{Python}
		
		\begin{alertblock}{Achtung!}
		Se utilizarán los mismos ejemplos que se pueden encontrar en la documentación	
		\end{alertblock}
	\end{frame}
	
\begin{frame}[fragile]{\texttt{lex.py}}
Suponga que está haciendo un lenguaje de programación y un usuario ingresa el siguiente string:

\begin{lstlisting}
           x = 3 + 42 * (s - t)
\end{lstlisting}

Un \textbf{tokenizer} o \emph{lexer} separará el string en \emph{tokens} uno por uno:

\begin{lstlisting}
'x','=','3','+','42','*','(','s','-','t',')'
\end{lstlisting}

Más específicamente, el \emph{string} será separado en pares de la forma \emph{(tipo, valor)}:

\begin{lstlisting}
('ID','x'), ('EQUALS','='), ('NUMBER','3'), 
('PLUS','+'), ('NUMBER','42'),('TIMES','*'),
('LPAREN','('), ('ID','s'), ('MINUS','-'),
('ID','t'), ('RPAREN',')'
\end{lstlisting}

\end{frame}
	
	\begin{frame}{{\texttt{lex.py}}}
		La identificación del tipo de un \emph{token} es típicamente definida por una serie de expresiones regulares. A continuación se mostrará como puede ser logrado con \texttt{lex.py} en \href{run:calclex.py}{\texttt{calclex.py}}
	\end{frame}
	
\begin{frame}[fragile]{Entendiendo \texttt{calclex.py}}
\justifying
Un \emph{lexer} debe proveer una lista de \emph{tokens} que definen todos los posibles tipos que deben ser reconocidos. El resultado del proceso de \emph{tokenization} es utilizado por \texttt{yacc.py}

\begin{example}
\begin{lstlisting}[language=Python]
tokens = (
   'NUMBER',
   'PLUS',
   'MINUS',
   'TIMES',
   'DIVIDE',
   'LPAREN',
   'RPAREN',
)
\end{lstlisting}			
\end{example}

\end{frame}

\begin{frame}{Entendiendo \texttt{calclex.py}}
	Cada \emph{token} es definido mediante una expresión regular compatible con el módulo \texttt{re} de \emph{Python}. Cada una de estas reglas están definidas en declaraciones con el prefijo especial \texttt{t\_}
\end{frame}

\begin{frame}[fragile]{Entendiendo \texttt{calclex.py}}
	Si la especificación de un \emph{token} es simple, es tan sencillo como declarar una variable de la forma \texttt{t\_<TOKEN> = <RE>}
	
\begin{example}
\begin{python}
t_PLUS = r'\+'
\end{python}
\end{example}

	En el ejemplo estamos diciendo que el \emph{token} \texttt{`PLUS'} está definido por la expresión regular \texttt{`\textbackslash+'}
\end{frame}

\begin{frame}[fragile]{Entendiendo \texttt{calclex.py}}
	Si la especificación de un \emph{token} es compleja, se puede declarar una función que trabaje con el \emph{token} extraído desde el \emph{input}. La función deberá llamarse \texttt{t\_<TOKEN>}, recibirá un argumento (el \emph{token} capturado) y retornará el \emph{token} con las modificaciones pertinentes. Es importante destacar que la expresión regular que define al \emph{token} es definida en el \emph{docstring} de la función
	
\begin{example}
\begin{python}
def t_NUMBER(t):
    r'\d+'
    t.value = int(t.value)
    return t
\end{python}	
\end{example}

\begin{alertblock}{Importante!}
El \emph{token} debe ser retornado o será descartado
\end{alertblock}

\end{frame}

\begin{frame}{Lea la documentación!}
\begin{itemize}
	\item Internamente \texttt{lex.py} usa la \emph{flag} \texttt{re.VERBOSE} en el uso el uso del módulo \texttt{re}. Esto significa que se aceptan comentarios con el símbolo \texttt{\#} y los espacios se descartan, por lo que en caso que su expresión regular necesite indicar explícitamente uno de estos caracteres, debe indicarlos como \texttt{[\#]} y \texttt{\textbackslash s} respectivamente

	\item Revise en la documentación la manera apropiada de definir palabras reservadas
\end{itemize}

\end{frame}

\end{document}











